% --------------------------------------------------------------
% This is all preamble stuff that you don't have to worry about.
% Head down to where it says "Start here"
% --------------------------------------------------------------
 
\documentclass[12pt]{article}\usepackage[]{graphicx}\usepackage[]{color}
%% maxwidth is the original width if it is less than linewidth
%% otherwise use linewidth (to make sure the graphics do not exceed the margin)
\makeatletter
\def\maxwidth{ %
  \ifdim\Gin@nat@width>\linewidth
    \linewidth
  \else
    \Gin@nat@width
  \fi
}
\makeatother

\definecolor{fgcolor}{rgb}{0.345, 0.345, 0.345}
\newcommand{\hlnum}[1]{\textcolor[rgb]{0.686,0.059,0.569}{#1}}%
\newcommand{\hlstr}[1]{\textcolor[rgb]{0.192,0.494,0.8}{#1}}%
\newcommand{\hlcom}[1]{\textcolor[rgb]{0.678,0.584,0.686}{\textit{#1}}}%
\newcommand{\hlopt}[1]{\textcolor[rgb]{0,0,0}{#1}}%
\newcommand{\hlstd}[1]{\textcolor[rgb]{0.345,0.345,0.345}{#1}}%
\newcommand{\hlkwa}[1]{\textcolor[rgb]{0.161,0.373,0.58}{\textbf{#1}}}%
\newcommand{\hlkwb}[1]{\textcolor[rgb]{0.69,0.353,0.396}{#1}}%
\newcommand{\hlkwc}[1]{\textcolor[rgb]{0.333,0.667,0.333}{#1}}%
\newcommand{\hlkwd}[1]{\textcolor[rgb]{0.737,0.353,0.396}{\textbf{#1}}}%

\usepackage{framed}
\makeatletter
\newenvironment{kframe}{%
 \def\at@end@of@kframe{}%
 \ifinner\ifhmode%
  \def\at@end@of@kframe{\end{minipage}}%
  \begin{minipage}{\columnwidth}%
 \fi\fi%
 \def\FrameCommand##1{\hskip\@totalleftmargin \hskip-\fboxsep
 \colorbox{shadecolor}{##1}\hskip-\fboxsep
     % There is no \\@totalrightmargin, so:
     \hskip-\linewidth \hskip-\@totalleftmargin \hskip\columnwidth}%
 \MakeFramed {\advance\hsize-\width
   \@totalleftmargin\z@ \linewidth\hsize
   \@setminipage}}%
 {\par\unskip\endMakeFramed%
 \at@end@of@kframe}
\makeatother

\definecolor{shadecolor}{rgb}{.97, .97, .97}
\definecolor{messagecolor}{rgb}{0, 0, 0}
\definecolor{warningcolor}{rgb}{1, 0, 1}
\definecolor{errorcolor}{rgb}{1, 0, 0}
\newenvironment{knitrout}{}{} % an empty environment to be redefined in TeX

\usepackage{alltt}
 
\usepackage[margin=1in]{geometry} 
\usepackage{amsmath,amsthm,amssymb}
 \usepackage{amsfonts}
\newcommand{\N}{\mathbb{N}}
\newcommand{\Z}{\mathbb{Z}}
 
\newenvironment{theorem}[2][Theorem]{\begin{trivlist}
\item[\hskip \labelsep {\bfseries #1}\hskip \labelsep {\bfseries #2.}]}{\end{trivlist}}
\newenvironment{lemma}[2][Lemma]{\begin{trivlist}
\item[\hskip \labelsep {\bfseries #1}\hskip \labelsep {\bfseries #2.}]}{\end{trivlist}}
\newenvironment{exercise}[2][Exercise]{\begin{trivlist}
\item[\hskip \labelsep {\bfseries #1}\hskip \labelsep {\bfseries #2.}]}{\end{trivlist}}
\newenvironment{problem}[2][Problem]{\begin{trivlist}
\item[\hskip \labelsep {\bfseries #1}\hskip \labelsep {\bfseries #2.}]}{\end{trivlist}}
\newenvironment{question}[2][Question]{\begin{trivlist}
\item[\hskip \labelsep {\bfseries #1}\hskip \labelsep {\bfseries #2.}]}{\end{trivlist}}
\newenvironment{corollary}[2][Corollary]{\begin{trivlist}
\item[\hskip \labelsep {\bfseries #1}\hskip \labelsep {\bfseries #2.}]}{\end{trivlist}}
\newenvironment{BaseCase}[2][Base Case]{\begin{trivlist}
\item[\hskip \labelsep {\bfseries #1}\hskip \labelsep {\bfseries #2.}]}{\end{trivlist}}
\newenvironment{Answer}[2][Answer]{\begin{trivlist}
\item[\hskip \labelsep {\bfseries #1}\hskip \labelsep {\bfseries #2.}]}{\end{trivlist}}
\newenvironment{Sketch}[2][Sketch]{\begin{trivlist}
\item[\hskip \labelsep {\bfseries #1}\hskip \labelsep {\bfseries #2.}]}{\end{trivlist}}
\IfFileExists{upquote.sty}{\usepackage{upquote}}{}
\begin{document}
 
% --------------------------------------------------------------
%                         Start here
% --------------------------------------------------------------
 
\title{Homework 6}%replace X with the appropriate number
\author{Steve Bronder\\ %replace with your name
Statistical Inference} %if necessary, replace with your course title
 
\maketitle
 
\begin{exercise}{1}Let Y have a t-distribution with 9 degrees of freedom. Using (only) repeated samples from a standard normal distribution, find the following:

\begin{itemize}
  \item $P(Y<-1)$
    \item $P(-1<Y<1)$
      \item $P(Y>2)$
\end{itemize}
\end{exercise}


\begin{Answer}{1}

For this exercise a for loop is used that takes ten samples from the normal distribuion and calculates the t statistic over ten thousand iterations. We then use the \texttt{ecdf()} function to compute the emirical cumulative distribution for each probability.
\begin{knitrout}
\definecolor{shadecolor}{rgb}{0.969, 0.969, 0.969}\color{fgcolor}\begin{kframe}
\begin{alltt}
\hlstd{Y.t} \hlkwb{<-} \hlkwa{NULL}

\hlkwa{for} \hlstd{(i} \hlkwa{in} \hlnum{1}\hlopt{:}\hlnum{10000}\hlstd{) \{}
    \hlstd{Ynorm} \hlkwb{<-} \hlkwd{rnorm}\hlstd{(}\hlnum{10}\hlstd{)}

    \hlstd{Ynorm.mean} \hlkwb{<-} \hlkwd{mean}\hlstd{(Ynorm)}

    \hlstd{Ynorm.var} \hlkwb{<-} \hlkwd{var}\hlstd{(Ynorm)}

    \hlstd{Y.t[i]} \hlkwb{<-} \hlstd{(Ynorm.mean)}\hlopt{/}\hlstd{(Ynorm.var}\hlopt{/}\hlkwd{sqrt}\hlstd{(}\hlnum{10}\hlstd{))}
\hlstd{\}}

\hlcom{# Y < -1}
\hlkwd{ecdf}\hlstd{(Y.t)(}\hlopt{-}\hlnum{1}\hlstd{)}
\end{alltt}
\begin{verbatim}
## [1] 0.1808
\end{verbatim}
\begin{alltt}
\hlcom{# -1 < Y < 1}
\hlnum{1} \hlopt{-} \hlkwd{ecdf}\hlstd{(Y.t)(}\hlopt{-}\hlnum{1}\hlstd{)} \hlopt{-} \hlstd{(}\hlnum{1} \hlopt{-} \hlkwd{ecdf}\hlstd{(Y.t)(}\hlnum{1}\hlstd{))}
\end{alltt}
\begin{verbatim}
## [1] 0.6386
\end{verbatim}
\begin{alltt}
\hlcom{# Y > 2}
\hlnum{1} \hlopt{-} \hlkwd{ecdf}\hlstd{(Y.t)(}\hlnum{2}\hlstd{)}
\end{alltt}
\begin{verbatim}
## [1] 0.0618
\end{verbatim}
\end{kframe}
\end{knitrout}

\begin{itemize}
  \item $P(Y<-1)=17.72\%$
    \item $P(-1<Y<1)=64.27\%$
      \item $P(Y>2)=6.24\%$
\end{itemize}
\end{Answer}

\begin{exercise}{2}
Use the \texttt{pt()} command in \textbf{R} to confirm your answers for exercise 1. Then, state how each of the three probabilities would change (greater/less than) if the random variable $Y$ were replaced with:
\begin{itemize}
  \item a standard normal variable $Z$
  \item a $t$-distributed random variable $w$, with 4 degrees of freedom
\end{itemize}
\end{exercise}

\begin{Answer}{2} 

Checking answers
\begin{knitrout}
\definecolor{shadecolor}{rgb}{0.969, 0.969, 0.969}\color{fgcolor}\begin{kframe}
\begin{alltt}
\hlcom{# }
\hlcom{# Y < -1}
\hlkwd{pt}\hlstd{(}\hlopt{-}\hlnum{1}\hlstd{,}\hlnum{9}\hlstd{)}
\end{alltt}
\begin{verbatim}
## [1] 0.1717182
\end{verbatim}
\begin{alltt}
\hlcom{# -1 < Y < 1}
\hlkwd{pt}\hlstd{(}\hlopt{-}\hlnum{1}\hlstd{,}\hlnum{9}\hlstd{,}\hlkwc{lower.tail}\hlstd{=}\hlnum{FALSE}\hlstd{)}\hlopt{-}\hlkwd{pt}\hlstd{(}\hlnum{1}\hlstd{,}\hlnum{9}\hlstd{,}\hlkwc{lower.tail}\hlstd{=}\hlnum{FALSE}\hlstd{)}
\end{alltt}
\begin{verbatim}
## [1] 0.6565636
\end{verbatim}
\begin{alltt}
\hlcom{# Y > 2}
\hlnum{1}\hlopt{-}\hlkwd{pt}\hlstd{(}\hlnum{2}\hlstd{,}\hlnum{9}\hlstd{)}
\end{alltt}
\begin{verbatim}
## [1] 0.03827641
\end{verbatim}
\end{kframe}
\end{knitrout}
These answers are close the answers received from the resampling method. Next, $Z$ is subsituted into $Y$ in order to see the change.

\begin{knitrout}
\definecolor{shadecolor}{rgb}{0.969, 0.969, 0.969}\color{fgcolor}\begin{kframe}
\begin{alltt}
\hlcom{# Standard normal random variable Z}
\hlcom{# Z < -1}
\hlkwd{pnorm}\hlstd{(}\hlopt{-}\hlnum{1}\hlstd{)}
\end{alltt}
\begin{verbatim}
## [1] 0.1586553
\end{verbatim}
\begin{alltt}
\hlcom{# -1 < Z < 1}
\hlkwd{pnorm}\hlstd{(}\hlopt{-}\hlnum{1}\hlstd{,}\hlkwc{lower.tail}\hlstd{=}\hlnum{FALSE}\hlstd{)}\hlopt{-}\hlkwd{pnorm}\hlstd{(}\hlnum{1}\hlstd{,}\hlkwc{lower.tail}\hlstd{=}\hlnum{FALSE}\hlstd{)}
\end{alltt}
\begin{verbatim}
## [1] 0.6826895
\end{verbatim}
\begin{alltt}
\hlcom{# Z > 2}
\hlnum{1}\hlopt{-}\hlkwd{pnorm}\hlstd{(}\hlnum{2}\hlstd{)}
\end{alltt}
\begin{verbatim}
## [1] 0.02275013
\end{verbatim}
\end{kframe}
\end{knitrout}
The tail probabilites grow smaller and the within probabilities grow larger. This is because a standard normal random variable is not as fat at the tails as the t distribution. Now, $w$ is substituted into $Y$ in order to see the change.
\begin{knitrout}
\definecolor{shadecolor}{rgb}{0.969, 0.969, 0.969}\color{fgcolor}\begin{kframe}
\begin{alltt}
\hlcom{# t distribution random variable w, with 4 degrees of freedom}

\hlcom{# W < -1}
\hlkwd{pt}\hlstd{(}\hlopt{-}\hlnum{1}\hlstd{,}\hlnum{4}\hlstd{)}
\end{alltt}
\begin{verbatim}
## [1] 0.1869505
\end{verbatim}
\begin{alltt}
\hlcom{# -1 < w < 1}
\hlkwd{pt}\hlstd{(}\hlopt{-}\hlnum{1}\hlstd{,}\hlnum{4}\hlstd{,}\hlkwc{lower.tail}\hlstd{=}\hlnum{FALSE}\hlstd{)}\hlopt{-}\hlkwd{pt}\hlstd{(}\hlnum{1}\hlstd{,}\hlnum{4}\hlstd{,}\hlkwc{lower.tail}\hlstd{=}\hlnum{FALSE}\hlstd{)}
\end{alltt}
\begin{verbatim}
## [1] 0.626099
\end{verbatim}
\begin{alltt}
\hlcom{# Y > 2}
\hlnum{1}\hlopt{-}\hlkwd{pt}\hlstd{(}\hlnum{2}\hlstd{,}\hlnum{4}\hlstd{)}
\end{alltt}
\begin{verbatim}
## [1] 0.05805826
\end{verbatim}
\end{kframe}
\end{knitrout}
The tail probabilites grew fatter. This is because as the number of degrees of freedom increases, the tails grow thinner.

\begin{exercise}{3} State clearly the null and alternative hypothesis, calculate the test statistic, and report the p-value using \textbf{R}
\end{exercise}

\begin{Answer}{3}

To test the claim that the population variance is greater than 6.2, we will perform a $\chi^{2}$ with a null of the population variance being less than orequal to 6.2 and an alternative that it is greater than 6.2

\begin{align}
H_0&:\sigma^{2}\leq{6.2}\\
H_A&:\sigma^{2}>6.2
\end{align}

\begin{knitrout}
\definecolor{shadecolor}{rgb}{0.969, 0.969, 0.969}\color{fgcolor}\begin{kframe}
\begin{alltt}
\hlstd{((}\hlnum{18}\hlopt{-}\hlnum{1}\hlstd{)}\hlopt{*}\hlnum{6.5}\hlstd{)}\hlopt{/}\hlnum{6.2}
\end{alltt}
\begin{verbatim}
## [1] 17.82258
\end{verbatim}
\begin{alltt}
\hlkwd{pchisq}\hlstd{(}\hlnum{17.82258}\hlstd{,}\hlnum{17}\hlstd{,}\hlkwc{lower.tail}\hlstd{=}\hlnum{FALSE}\hlstd{)}
\end{alltt}
\begin{verbatim}
## [1] 0.4001157
\end{verbatim}
\end{kframe}
\end{knitrout}
With a p-value of .4 and $\alpha=.01$ we do not reject the null and conclude that the popluation variance is less than or equal to 6.2.
\end{Answer}

\begin{exercise}{4} State clearly the null and alternative hypothesis, calculate the test statistic, and report the p-value using \textbf{R}.
\end{exercise}

\begin{Answer}{4}
To test the claim that the there is a difference in the standard deviations, we will perform a $f$ test with a null of the standard deviation of fast food one ($sd_1$) and standard deviation of fast food two ($sd_2$) being the same an alternate that they are different.

\begin{align}
H_0&:sd_1=sd_2\\
H_A&:sd_1\neq{sd_2}
\end{align}

\begin{knitrout}
\definecolor{shadecolor}{rgb}{0.969, 0.969, 0.969}\color{fgcolor}\begin{kframe}
\begin{alltt}
\hlnum{4.8}\hlopt{/}\hlnum{3.5}
\end{alltt}
\begin{verbatim}
## [1] 1.371429
\end{verbatim}
\begin{alltt}
\hlkwd{pf}\hlstd{(}\hlnum{1.371429}\hlstd{,}\hlnum{12}\hlstd{,}\hlnum{10}\hlstd{)}\hlopt{/}\hlnum{2}
\end{alltt}
\begin{verbatim}
## [1] 0.3435938
\end{verbatim}
\end{kframe}
\end{knitrout}
With a p-value of .34 and $\alpha=.01$ we do not reject the null and conclude the standard deviations are equal.


\end{Answer}
\begin{exercise}{5} State clearly the null and alternative hypothesis, calculate the test statistic, and report the p-value using \textbf{R}.
\end{exercise}

\begin{Answer}{5}
To test the claim that the average single-person household received at least thirty seven phone calls per month, we will perform a $t$ test with a null of the sample mean ($\bar{X}$) being equal to the population mean ($\mu$) and an alternative that they are not equal.

\begin{align}
H_0&:\bar{X}=\mu\\
H_A&:\bar{X}\neq{\mu}
\end{align}

\begin{knitrout}
\definecolor{shadecolor}{rgb}{0.969, 0.969, 0.969}\color{fgcolor}\begin{kframe}
\begin{alltt}
\hlcom{# Exercise 5}

\hlstd{(}\hlnum{34.9}\hlopt{-}\hlnum{37}\hlstd{)}\hlopt{/}\hlstd{(}\hlnum{6}\hlopt{/}\hlkwd{sqrt}\hlstd{(}\hlnum{29}\hlstd{))}
\end{alltt}
\begin{verbatim}
## [1] -1.884808
\end{verbatim}
\begin{alltt}
\hlkwd{pt}\hlstd{(}\hlopt{-}\hlnum{1.88}\hlstd{,}\hlnum{28}\hlstd{)}\hlopt{/}\hlnum{2}
\end{alltt}
\begin{verbatim}
## [1] 0.0176375
\end{verbatim}
\end{kframe}
\end{knitrout}
With a p-value of .017 and $\alpha=.05$, we reject the null and conclude that there is a difference in the sample mean and population mean.
% ----------------------
%----------------------------------------
%     You don't have to mess with anything below this line.
% --------------------------------------------------------------
 
\end{document}
